\documentclass{scrartcl}    % To have more options like subtitle etc.
\usepackage[T1]{fontenc}    % Make sure german "Umlaute" and "ß" are copied properly from result .pdf
%\graphicspath{{images/}}    % Default image path
\usepackage[utf8]{inputenc}
\usepackage{amsmath,amssymb}
\usepackage{graphicx}
\usepackage{hyperref}
\usepackage{makeidx}
\usepackage{mathrsfs}
\usepackage{nomencl}
\usepackage{braket}       % or \usepackage{physics}

\usepackage[backend=biber,
            style=authoryear,
            hyperref=true,
            backref=true]{biblatex}
\addbibresource{LV_BA.bib} 
\setlength\bibitemsep{1em} % Adjust the spacing as needed
\usepackage{setspace}       % For line spacing
\onehalfspacing             % Set line spacing to 1.5x
\usepackage[a4paper,        % Adjust margins
            left=2.5cm,     % Left margin
            right=2.5cm,    % Right margin
            top=2.5cm,      % Top margin
            bottom=4.0cm]{geometry} % Bottom margin


\DeclareCiteCommand{\cite}
  {\usebibmacro{prenote}}
  {\bibhyperref{\printnames{labelname}~(\printdate)}}
  {\multicitedelim}
  {\usebibmacro{postnote}}

% Optionally patch \parencite as well
\DeclareCiteCommand{\parencite}
  {\usebibmacro{prenote}}
  {\bibhyperref{(\printnames{labelname},~\printdate)}}
  {\multicitedelim}
  {\usebibmacro{postnote}}% Index setup

\AtEveryCitekey{\clearfield{month}}% 

\usepackage{bm}
\usepackage{pdfpages}       % Include whole pdfs
\usepackage{siunitx}        % Formatting SI units
% Default per-mode
\sisetup{
  per-mode=fraction,
  separate-uncertainty = true,
  table-align-uncertainty = true,
  detect-weight = true,
  detect-family = true
}
%\sisetup{locale = DE}       % Correct decimal marker etc.
\DeclareSIUnit\litre{l}     % Lowercase l for litre
\hypersetup{
  colorlinks   = true,    % Colours links instead of ugly boxes
  urlcolor     = blue,    % Colour for external hyperlinks
  linkcolor    = blue,    % Colour of internal links
  citecolor    = blue      % Colour of citations
}
\usepackage{multirow}     % Multirow tables
\usepackage{float}        % Move figures with [H]
\usepackage[version=4]{mhchem} % Chemical formulas
\usepackage{amsfonts}     % Math fonts
\usepackage{enumerate}
\usepackage{booktabs}
\usepackage{xcolor}

\usepackage{fancyhdr}

\newcommand{\studentname}{Lukas Riethm\"uller}
\newcommand{\studentid}{2025403370}
\newcommand{\hwNumber}{1}

\pagestyle{fancy}
\fancyhf{}                             % clear header/footer
\lhead{Embodied AI\\ Homework Nr. \hwNumber}       % top-left
\rhead{\studentname\\ \studentid}     % top-right (two lines)
\renewcommand{\headrulewidth}{0.4pt}   % header rule thickness
\setlength{\headheight}{14pt}          

%\makeindex

%Später wieder anschalten

%\usepackage{lastpage}
%\usepackage{scrlayer-scrpage}
%\cfoot{\thepage\ von \pageref*{LastPage}} % Seite x von y; * entfernt Hyperlink

\begin{document}    

\section*{Setting up}

\begin{figure}[H]
  \centering
  \includegraphics[width=0.8\textwidth]{renders/render0/00000.png}
  \caption{Initial render of the reconstructed mesh.}
  \label{fig:initial-render}
\end{figure}
%include picture from intitial render

\section{Preperation (Point Clouds)}

Generating the point cloud from the files in fruit.zip proves to be a very long task. The pipeline was as follows: first the feature extraction happens (this took 13 minutes in google colab), i.e. the program finds the interesting features in the individual images. Then the feature matching begins (google colab was too slow to finish this part and I was not able to follow through), where features across different images are identified, linking their matches. There should also be a last step (I could not get to with google colab), where the program attempts to reconstruct the original object from the now matched features in the form of a mesh.\\
The next task is to visualize an already processed mesh with open3D, the results are seen in  figure \ref{fig:mesh}:

\begin{figure}[H]
  \centering
  \includegraphics[width=0.8\textwidth]{results/pointcloud_1.jpg}
  \caption{Processed mesh with open3D}
  \label{fig:mesh}
\end{figure}

%include screenshot

\section{Building Gaussian Primitives}

A general 3D Gaussian is defined thorugh its mean \(\boldsymbol\mu\) and covariance \(\Sigma\).
If we obtain an ellipsoid by applying a linear map \(A\) to an isotropic unit Gaussian,
i.e. \(\mathbf{y}=\boldsymbol\mu + A\mathbf{x}\) with \(\mathbf{x}\sim\mathcal{N}(\mathbf{0},I)\),
then the covariance of \(\mathbf{y}\) is
\[
\Sigma = A A^{T}.
\]

Let \(S=\operatorname{diag}(s_x,s_y,s_z)\) be a scaling matrix and
let \(R\in\mathrm{SO}(3)\) be a rotation matrix. Suppose
\(A=R\,S\), then we get
\[
\Sigma \;=\; (R S)(R S)^{T}
    \;=\; R S S^{T} R^{T}
    \;=\; R\,S^{2}\,R^{T},
\]
where \(S^{2}=\operatorname{diag}(s_x^{2},s_y^{2},s_z^{2})\). If we assume general gaussian with coveriance $\Sigma_0$, then
\[
\Sigma \;=\; R S \Sigma_0 S^{T} R^{T}.
\] \\

A viewing transformation $\mathbf{T}_v$ is a 4x4 matrix a rotation part $A_v\in\mathbb{R}^{3\times3}$ and translation $\mathbf{t}_v\in\mathbb{R}^3$,
\[
\mathbf{T}_v=\begin{pmatrix}A_v & \mathbf{t}_v\\[2pt]\mathbf{0}^T & 1\end{pmatrix},\qquad
\mathbf{y}=A_v\mathbf{x}+\mathbf{t}_v.
\]
Then
\[
\mathcal{N}(\boldsymbol\mu,\Sigma)\xrightarrow{\mathbf{T}_v}
\mathcal{N}\bigl(A_v\boldsymbol\mu+\mathbf{t}_v,\;A_v\,\Sigma\,A_v^{T}\bigr).
\]
The translation only shifts the mean, while the rotation effects both covariance and mean.


For a nonlinear projection transformation $\pi(\mathbf{x})$, linearize it at the mean $\boldsymbol\mu$ with Jacobian $J=\left.\frac{\partial\pi}{\partial\mathbf{x}}\right|_{\boldsymbol\mu}$
\[
\pi(x)\approx J(\mu)(x-\mu) + \pi(\mu).
\]
This linearly transforms our covariance as follows:
\[
\Sigma_{\text{proj}}\approx J\,\Sigma\,J^{T}.
\]


We now want to find a square bounding-box side length that cover 99\% of the confidence region. Its probability level for a gaussian is defined through $\chi^2$ quantile
\[
(x-\mu)^T\Sigma_{\text{proj}}^{-1}(x-\mu) = \chi^2_{2;0.99},\qquad
\chi^2_{2;0.99}\approx 9.21034037197618,\qquad
k:=\sqrt{\chi^2_{2;0.99}}\approx 3.034854
\]
Assuming our Covariance is diagonalized (else do so), we can express the semi-axis lengths as follows:
\[
\max_{(x-\mu)^T\Sigma^{-1}(x-\mu)\le k^2} |x_x-\mu_x| = k\sqrt{\Sigma_{11}},\qquad
\max_{(x-\mu)^T\Sigma^{-1}(x-\mu)\le k^2} |x_y-\mu_y| = k\sqrt{\Sigma_{22}}
\]
To have a confident cover use the largest semi axis for the squared cover:
\[
\ell_{\text{square}} \;=\; 2k\;\max\!\bigl(\sqrt{\Sigma_{11}},\sqrt{\Sigma_{22}}\bigr)
    \;\approx\; 2\cdot 3.034854\;\max(\sigma_x,\sigma_y)
    \;\approx\; 6.069708\;\max(\sigma_x,\sigma_y),
\]
where \(\sigma_x=\sqrt{\Sigma_{11}}\) and \(\sigma_y=\sqrt{\Sigma_{22}}\).

\section{Better Rendering}

In figure \ref{fig:renders-comparison} we can see the results of the rendering with 0th and 1st degree spherical harmonics. I cannot make out a difference with my eyes, possibly I made a mistake when implementing the gaussian splatting (I did flag sh\_degree 1).

\begin{figure}[H]
  \centering
  \begin{minipage}[b]{0.48\textwidth}
    \centering
    \includegraphics[width=\linewidth]{renders/render2.4/00000.png}
    \vspace{0.5em}

    {\small (a) SH of degree 0}
  \end{minipage}\hfill
  \begin{minipage}[b]{0.48\textwidth}
    \centering
    \includegraphics[width=\linewidth]{renders/render 3.1/00000.png}
    \vspace{0.5em}

    {\small (b) SH of degree 1}
  \end{minipage}

  \vspace{0.8em}

  \begin{minipage}[b]{0.48\textwidth}
    \centering
    \includegraphics[width=\linewidth]{renders/render2.4/00002.png}
    \vspace{0.5em}

    {\small (c) SH of degree 0}
  \end{minipage}\hfill
  \begin{minipage}[b]{0.48\textwidth}
    \centering
    \includegraphics[width=\linewidth]{renders/render 3.1/00002.png}
    \vspace{0.5em}

    {\small (d) SH of degree 1}
  \end{minipage}

  \caption{Left column with 0th degree spherical harmonics (SH), right column with 1st degree SH}
  \label{fig:renders-comparison}
\end{figure}

In the next step we implement opacity-based-pruning, the resulting effects vor various opacity threshholds can be seen in figure \ref{fig:opacity-renders}. While a minimum opacity of 0.01 has no visual difference (roughly 8\% of gaussians were filtered), 0.1 starts removing significant amounts that do not introduce noticeable defects (36\% of gaussians filtered), while values of 0.5 and higher introduce considerable defects.


\begin{figure}[H]
  \centering
  \begin{minipage}[b]{0.48\textwidth}
    \centering
    \includegraphics[width=\linewidth]{renders/Opacity .999 true/00001.png}
    \vspace{0.5em}

    {\small (a) Minimal opacity of 0.999}
  \end{minipage}\hfill
  \begin{minipage}[b]{0.48\textwidth}
    \centering
    \includegraphics[width=\linewidth]{renders/Opacity 0.5 true/00001.png}
    \vspace{0.5em}

    {\small (b) Minimal opacity of 0.5}
  \end{minipage}

  \vspace{0.8em}

  \begin{minipage}[b]{0.48\textwidth}
    \centering
    \includegraphics[width=\linewidth]{renders/Opacity 0.1 true/00001.png}
    \vspace{0.5em}

    {\small (c) Minimal opacity of 0.1}
  \end{minipage}\hfill
  \begin{minipage}[b]{0.48\textwidth}
    \centering
    \includegraphics[width=\linewidth]{renders/Opacity 0.01 true/00001.png}
    \vspace{0.5em}

    {\small (d) Minimal opacity of 0.01}
  \end{minipage}

  \caption{Renders with varying opacity levels (top-left: 0.999, top-right: 0.5, bottom-left: 0.1, bottom-right: 0.01).}
  \label{fig:opacity-renders}
\end{figure}


\end{document}
